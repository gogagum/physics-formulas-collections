\documentclass{article}
\usepackage[utf8]{inputenc}
\usepackage[T1]{fontenc}
\usepackage[left=0cm,right=0cm,top=0cm,bottom=0cm,bindingoffset=0cm]{geometry}
\usepackage[russian]{babel}
\usepackage{amssymb,amsmath}

\begin{document}
\begin{tabular}{ |p{4.1cm}|p{5.4cm}|p{3.9cm}|p{5.6cm}|  }
\hline
\multicolumn{4}{|c|}{Оптика.} \\
\hline
Закон преломления:                                                           &  % [1, 1]
$n_1 \sin{\varphi_1} = n_2 \sin{\varphi_2}$                                  &  % [1, 2]
Электромагнитное поле сферической волны:                                     &  % [1, 3]
$\begin{aligned}
\vec{H} = k^2 (\vec{n} \times \vec{p_0}) \frac{e^{i(kr - \omega t)}}{r}, \\
\vec{E} = \vec{H} \times \vec{n}, k = \frac{\omega}{c},
\vec{n} = \frac{\vec{r}}{r}
\end{aligned}$                 
\vspace{5pt}                                                                 \\ % [1, 4]
\hline
Закон отражения:                                                             &  % [2, 1]
Угол падения равен углу отражения.                                           &  % [2, 2]
Показатели преломления и отражения:                                          &  % [2, 3]
$\sqrt{\varepsilon} = n + i\kappa$                                           \\ % [2, 4]
\hline
Формула тонкой линзы:                                                        &  % [3, 1]
$\frac{1}{F} = \frac{1}{f} + \frac{1}{d}$                                    &  % [3, 2]
Асимптотическое значение показателя преломления:                             &  % [3, 3]
$n_0 = n(0) = \sqrt{1 + \frac{\omega_p^2}{\omega_0^2}}$                      \\ % [3, 4]
\hline
Фокусное расстояние через радиусы кривизны:                                  &  % [4, 1]
$\frac{1}{F} = (n-1)\left(\frac{1}{R_1} + \frac{1}{R_2}\right)$ (<<->> перед
$\frac{1}{R_i}$, если соответствующая поверхность вогнутая.)                 &  % [4, 2]
Закон Бугера:                                                                &  % [4, 3]
$\begin{aligned}
k = \frac{\omega \sqrt{\varepsilon}}{c} =
 \frac{\omega}{c}n + i \frac{\omega}{c} \kappa =    \\
 k_r + i \frac{\alpha}{2},
I = I_0 e^{-\alpha x}
\end{aligned}$                                                               \\ % [4, 4]
\hline
Фокусное расстояние двух линз:                                               &  % [5, 1]
$\frac{1}{F} = \frac{1}{F_1} + \frac{1}{F_2} - \frac{d}{F_1 F_2}$,
где $d$ -- расстояние между линзами.                                         &  % [5, 2]
Диэлектрическая проницаемость плазмы:                                        &  % [5, 3]
$\varepsilon = 1 - (\frac{\omega_p}{\omega})^2 =
 1 - \frac{4 \pi N e^2}{m \omega^2}$                                         \\ % [5, 4]
\hline
Волновое уравнение:                                                          &  % [6, 1]
$\frac{1}{v^2} \frac{\partial^2 \vec{E}}{\partial t^2} - \Delta \vec{E} = 0,
\frac{1}{v^2} \frac{\partial^2 \vec{H}}{\partial t^2} - \Delta \vec{H} = 0$  &  % [6, 2]
Групповая скорость:                                                          &  % [6, 3]
$v_{gr} = \frac{d \omega}{d k}$                                                 \\ % [6, 4]
\hline
Скорость света в среде:                                                      &  % [7, 1]
$v = \frac{c}{n} = \frac{c}{\sqrt{\varepsilon\mu}}$                          &  % [7, 2]
Интенсивность суммы двух волн:                                               &  % [7, 3]
$I = I_1 + I_2 + 2 \sqrt{I_1 I_2} cos [\Delta \varphi(r)]$                                       \\ % [7, 4]
\hline
Уравнение Гельмгольца:                                                       &  % [8, 1]
$\Delta \vec{E} + \frac{\omega^2}{v^2} \vec{E} = 0,
\Delta \vec{H} + \frac{\omega^2}{v^2} \vec{H} = 0$                           &  % [8, 2]
Видимость:                                                                   &  % [8, 3]
$V = \frac{I_{max} - I_{min}}{I_{max} + I_{min}}$                                                     \\ % [8, 4]
\hline
Плоская волна:                                                               &  % [9, 1]
$\vec{E}(x, t) = \vec{E_1} \cos((\vec{k}, \vec{x}) - \omega t + \varphi_1)$  &  % [9, 2]
Интенсивность максимумов:                                                    &  % [9, 3]
$I_{max} = (\sqrt{I_1} + \sqrt{I_2})^2$                                      \\ % [9, 4]
\hline
Комплексная амплитуда:                                                       &  % [10, 1]
$\vec{E}(x, t) = \vec{E_0} \exp(i(\vec{k}, \vec{x}) - \omega t)$             &  % [10, 2]
Интенсивность минимумов:                                                     &  % [10, 3]
$I_{min} = (\sqrt{I_1} - \sqrt{I_2})^2$                                      \\ % [10, 4]
\hline
Волновое число:                                                              &  % [11, 1]
$|\vec{k}| = \frac{\omega}{v} = \frac{\omega}{c} n, \vec{k}$ задаёт
 направление распространения волны.                                          &  % [11, 2]
Видимость суммы двух волн:                                                   &  % [11, 3]
$V = \frac{2 \sqrt{I_1 I_2}}{I_1 + I_2}$                                     \\ % [11, 4]
\hline
Фазовая скорость волны:                                                      &  % [12, 1]
$v = \frac{\omega}{k} = \frac{c}{n}$                                         &  % [12, 2]
Интер-ция двух плоских волн с уравнениями:
$A_i(\vec{r}, t) = a_i \cos(\vec{k_i}\vec{r} - \omega t + \varphi_i),
 i = 1, 2$                                                                   &  % [12, 3]
$\begin{aligned}
I = I_1 + I_2 + 2 \sqrt{I_1 I_2} cos [\vec{K} \vec{r} + \delta],  \\
\vec{K} = \vec{k_1} - \vec{k_2}, \delta = \varphi_1 - \varphi_2
\end{aligned}$                                                               \\ % [12, 4]
\hline
Длина волны:                                                                 &  % [13, 1]
$\lambda = vT = \frac{C}{nV} = \frac{2 \pi c}{n \omega} =
 \frac{\lambda_0}{n}, \lambda_0$ -- длина волны в вакууме.                   &  % [13, 2]
Условие максимумов:                                                          &  % [13, 3]
$(\vec{K} \cdot \vec{r}) + \delta = 2 \pi m, m \in \mathbb{Z}$                          \\ % [13, 4]
\hline
Фаза волны:                                                                  &  % [14, 1]
$\varphi = (\vec{k}, \vec{r}) - \omega t$                                    &  % [14, 2]
Условие минимумов:                                                           &  % [14, 3]
$(\vec{K} \cdot \vec{r}) + \delta = \pi (2 m + 1), m \in \mathbb{Z}$         \\ % [14, 4]
\hline
Связь амплитуд $\vec{H}$ и $\vec{E}$:                                        &  % [15, 1]
$\sqrt{\varepsilon} E_0 = \sqrt{\mu} H_0$                                    &  % [15, 2]
Расстояние между полосами:                                                   &  % [15, 3]
$\begin{aligned}
K = |\vec{k_1} - \vec{k_2}| = 2k\sin(\frac{\alpha}{2}),  \\
\Delta x = \frac{2 \pi}{K} = \frac{\lambda}{2 \sin(\frac{\alpha}{2})}
\end{aligned}$                                                               \\ % [15, 4]
\hline
Уравнения Максвелла для плоских волн:                                        &  % [16, 1]
$\begin{aligned}
\vec{k} \times \vec{E} = \frac{\omega}{c} \vec{B}, (\vec{k}, \vec{D}) = 0, \\
\vec{k} \times \vec{H} = \frac{\omega}{c} \vec{D}, (\vec{k}, \vec{B}) = 0
\end{aligned}$                                                               &  % [16, 2]
Максимумы в схеме Юнга:                                                      &  % [16, 3]
$x_{max} = \frac{\lambda L}{d} m$                                            \\ % [16, 4]
\hline
Расходящаяся сферическая волна:                                              &  % [17, 1]
$A = A_0 \frac{e^{i k r-i \omega t}}{r}$                                     &  % [17, 2]
Минимумы в схеме Юнга:                                                       &  % [17, 3]
$x_{min} = \frac{\lambda L}{d} (m + \frac{1}{2})$                                                                             \\ % [17, 4]
\hline
Сходящаяся сферическая волна:                                                &  % [18, 1]
$A = A_0 \frac{e^{-i k r-i \omega t}}{r}$                                    &  % [18, 2]
Функция временной когерентности:                                             &  % [18, 3]
$\begin{aligned}
\overline{A^2(t)} = \overline{A^2(t+\tau)} = I_0, \\
\overline{A^2(t) A^2(t+\tau)} = \Gamma(\tau),     \\
I = 2 I_0 + 2 \Gamma(\tau)
\end{aligned}$                                                               \\ % [18, 4]
\hline
\end{tabular}

\newpage

\begin{tabular}{ |p{4.2cm}|p{5.3cm}|p{6cm}|p{3.5cm}|  }
\hline
\multicolumn{4}{|c|}{Оптика.} \\
\hline
Комплексная функция когерентности:                                           &  % [1, 1]
$\hat{\Gamma}(\tau) = I_0 e^{i \omega_0 \tau}$                               &  % [1, 2]
                                                                             &  % [1, 3]
                                                                             \\ % [1, 4]
\hline
Степень временной когерентности $\gamma$:                                    &  % [2, 1]
$\begin{aligned}
\hat{\Gamma}(\tau) = I_0 \hat{\gamma}(\tau),  \\
\hat{\gamma}(\tau) = | \gamma (\tau) | e^{i[\omega_0 \tau + \varphi_0(\tau)]}
\end{aligned}$                                                               &  % [2, 2]
                                                                             &  % [2, 3]
                                                                             \\ % [2, 4]
\hline
Связь видимости и степени когерентности:                                     &  % [3, 1]
$V = |\gamma(\tau)|$                                                         &  % [3, 2]
                                                                             &  % [3, 3]
                                                                             \\ % [3, 4]
\hline
Функция временной когерентности:                                             &  % [4, 1]
$\Gamma(\tau) = \frac{1}{\Delta \tau}
 \int_0^{\Delta \tau} A(t_1) A(t_1 + \tau) dt$                               &  % [4, 2]
                                                                             &  % [4, 3]
                                                                             \\ % [4, 4]
\hline
Теорема Винера-Хинчина:                                                      &  % [5, 1]
$\begin{aligned}
dI_0 = J(\omega) d\omega, I_0 = \int_0^\infty J(\omega) d\omega, \\
I = 2 \int_0^\infty J(\omega)(1 + \cos(\omega \tau)) d\omega,    \\
\Gamma(\tau) = \int_0^{\infty} J(\omega) \cos{\omega \tau} d \omega
\end{aligned}$                                                               &  % [5, 2]
                                                                             &  % [5, 3]
                                                                             \\ % [5, 4]
\hline
Теорема Винера-Хинчина (комплексная форма):                                  &  % [6, 1]
$\hat{\Gamma}(\tau) =
 \int_{0}^{\infty} J(\omega) e^{i \omega \tau} d\omega$                      &  % [6, 2]
                                                                             &  % [6, 3]
                                                                             \\ % [6, 4]
\hline
Радиус $m$-ой зоны Френеля:                                                  &  % [7, 1]
$R_m = \sqrt{m \lambda f}$, где $f$ - расстояние от отверстия до экрана.     &  % [7, 2]
                                                                             &  % [7, 3]
                                                                             \\ % [7, 4]
\hline
П-дь $m$-ой зоны Френеля:                                                    &  % [8, 1]
$S_m = \pi m \lambda f$                                                      &  % [8, 2]
                                                                             &  % [8, 3]
                                                                             \\ % [8, 4]
\hline
Разность хода от двух соседних щелей в дифракционной решетке и
 условие (главных) максимумов:                                               &  % [9, 1]
$\begin{aligned}
\Delta = d \sin \Theta, \\
d \sin \Theta = m \lambda, \\
d(\sin{\Theta}-\sin{\Theta_0}) = m \lambda.
\end{aligned}$                                                               &  % [9, 2]
                                                                             &  % [9, 3]
                                                                             \\ % [9, 4]
\hline
Распределение интенсивности излучения при дифракции Фраунгофера на щели:     &  % [10, 1]
$I = I_0 \frac{sin^2(\frac{N}{2} kd \cdot sin(\Theta))}
              {sin^2(\frac{1}{2} kd \cdot sin(\Theta))}$                     &  % [10, 2]
                                                                             &  % [10, 3]
                                                                             \\ % [10, 4]
\hline
Распределение интенсивности излучения при дифракции Фраунгофера
 на круглом отверстии:                                                       &  % [11, 1]
$\begin{aligned}
I = I_0 \left(\frac{\pi R^2}{\lambda z}\right)^2
\left(\frac{2 J_1(qR)}{qR}\right)^2,              \\
q = k \sin{\Theta}
\end{aligned}$                                                               &  % [11, 2]
                                                                             &  % [11, 3]
                                                                             \\ % [11, 4]
\hline
Интенсивность излучения в главных максимумах (дифракционная решётка):        &  % [12, 1]
$I = N^2 I_0$                                                                &  % [12, 2]
                                                                             &  % [12, 3]
                                                                             \\ % [12, 4]
\hline
Угол первого минимума при дифракции на решетке:                              &  % [13, 1]
$L \sin{\Theta} = \lambda$                                                   &  % [13, 2]
                                                                             &  % [13, 3]
                                                                             \\ % [13, 4]
\hline
Угловая ширина главного максимума:                                           &  % [14, 1]
$\Delta \Theta = \frac{2 \lambda}{Nd}$                                       &  % [14, 2]
                                                                             &  % [14, 3]
                                                                             \\ % [14, 4]
\hline
Интенсивность в дополнительных максимумах:                                   &  % [15, 1]
$I^{(n)} = \frac{4 I_0}{(kd\Theta_{max})^2} =
 \frac{I_0 N^2}{\pi^2 (n + \frac{1}{2})^2}$                                  &  % [15, 2]
                                                                             &  % [15, 3]
                                                                             \\ % [15, 4]
\hline
Влияние ширины щели на дифракционную картину:                                &  % [16, 1]
$I=I_{0} \frac{\sin ^{2}\left(\frac{1}{2} k b \sin \theta\right)}
              {\left(\frac{1}{2} k b \sin \theta\right)^{2}}
         \frac{\sin ^{2}\left(\frac{1}{2} N k d \sin \theta\right)}
              {\sin ^{2}\left(\frac{1}{2} k d \sin \theta\right)}$           &  % [16, 2]
                                                                             &  % [16, 3]
                                                                             \\ % [16, 4]
\hline
Критерий Рэлея:                                                              &  % [17, 1]
$\Theta > 1,22 \frac{\lambda}{D}, l > 1,22 \frac{\lambda}{D} L$              &  % [17, 2]
                                                                             &  % [17, 3]
                                                                             \\ % [17, 4]
\hline
                                                                             &  % [18, 1]
                                                                             &  % [18, 2]
                                                                             &  % [18, 3]
                                                                             \\ % [18, 4]
\hline
\end{tabular}


\end{document}
