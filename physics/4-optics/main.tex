\documentclass{article}
\usepackage[utf8]{inputenc}
\usepackage[T1]{fontenc}
\usepackage[left=0cm,right=0cm,top=0cm,bottom=0cm,bindingoffset=0cm]{geometry}
\usepackage[russian]{babel}
\usepackage{amssymb,amsmath}

\begin{document}
\begin{tabular}{ |p{4.1cm}|p{5.4cm}|p{4.1cm}|p{4.4cm}|  }
\hline
\multicolumn{4}{|c|}{Оптика.} \\
\hline
Закон преломления:                                                           &  % [1, 1]
$n_1 \sin{\varphi_1} = n_2 \sin{\varphi_2}$                                  &  % [1, 2]
Электромагнитное поле сферической волны:                                     &  % [1, 3]
$\begin{aligned}
\vec{H} = k^2 (\vec{n} \times \vec{p_0}) \frac{e^{i(kr - \omega t)}}{r}, \\
\vec{E} = \vec{H} \times \vec{n}, k = \frac{\omega}{c},
\vec{n} = \frac{\vec{r}}{r}
\end{aligned}$                                                               \\ % [1, 4]
\hline
Закон отражения:                                                             &  % [2, 1]
Угол падения равен углу отражения.                                           &  % [2, 2]
Показатели преломления и отражения:                                          &  % [2, 3]
$\sqrt{\varepsilon} = n + i\kappa$                                           \\ % [2, 4]
\hline
Формула тонкой линзы:                                                        &  % [3, 1]
$\frac{1}{F} = \frac{1}{f} + \frac{1}{d}$                                    &  % [3, 2]
Асимптотическое значение показателя преломления:                             &  % [3, 3]
$n_0 = n(0) = \sqrt{1 + \frac{\omega_p^2}{\omega_0^2}}$                      \\ % [3, 4]
\hline
Фокусное расстояние через радиусы кривизны:                                  &  % [4, 1]
$\frac{1}{F} = (n-1)(\frac{1}{R_1} + \frac{1}{R_2})$ (<<->> перед
$\frac{1}{R_i}$, если соответствующая поверхность вогнутая)                  &  % [4, 2]
Закон Бугера:                                                                &  % [4, 3]
$\begin{aligned}
k = \frac{\omega \sqrt{\varepsilon}}{c} =
 \frac{\omega}{c}n + i \frac{\omega}{c} \kappa =    \\
 k_r + i \frac{\alpha}{2},
I = I_0 e^{-\alpha x}
\end{aligned}$                                                                             \\ % [4, 4]
\hline
Фокусное расстояние двух линз:                                               &  % [5, 1]
$\frac{1}{F} = \frac{1}{F_1} + \frac{1}{F_2} - \frac{1}{F_1 F_2}$            &  % [5, 2]
Диэлектрическая проницаемость плазмы:                                        &  % [5, 3]
$\varepsilon = 1 - (\frac{\omega_p}{\omega})^2 =
 1 - \frac{4 \pi N e^2}{m \omega^2}$                                         \\ % [5, 4]
\hline
Волновое уравнение:                                                          &  % [6, 1]
$\frac{1}{v^2} \frac{\partial^2 \vec{E}}{\partial t^2} - \Delta \vec{E} = 0,
\frac{1}{v^2} \frac{\partial^2 \vec{H}}{\partial t^2} - \Delta \vec{H} = 0$  &  % [6, 2]
Групповая скорость:                                                          &  % [6, 3]
$v_{gr} = \frac{d \omega}{d k}$                                                 \\ % [6, 4]
\hline
Скорость света в среде:                                                      &  % [7, 1]
$v = \frac{c}{n} = \frac{c}{\sqrt{\epsilon\mu}}$                             &  % [7, 2]
                                                                             &  % [7, 3]
                                                                             \\ % [7, 4]
\hline
Уравнение Гельмгольца:                                                       &  % [8, 1]
$\Delta \vec{E} + \frac{\omega^2}{v^2} \vec{E} = 0,
\Delta \vec{H} + \frac{\omega^2}{v^2} \vec{H} = 0$                           &  % [8, 2]
                                                                             &  % [8, 3]
                                                                             \\ % [8, 4]
\hline
Плоская волна:                                                               &  % [9, 1]
$\vec{E}(x, t) = \vec{E_1} \cos((\vec{k}, \vec{x}) - \omega t + \varphi_1)$  &  % [9, 2]
                                                                             &  % [9, 3]
                                                                             \\ % [9, 4]
\hline
Комплексная амплитуда:                                                       &  % [10, 1]
$\vec{E}(x, t) = \vec{E_0} \exp(i(\vec{k}, \vec{x}) - \omega t)$             &  % [10, 2]
                                                                             &  % [10, 3]
                                                                             \\ % [10, 4]
\hline
Волновое число:                                                              &  % [11, 1]
$|\vec{k}| = \frac{\omega}{v} = \frac{\omega}{c} n, \vec{k}$ задаёт
 направление распространения волны.                                          &  % [11, 2]
                                                                             &  % [11, 3]
                                                                             \\ % [11, 4]
\hline
Фазовая скорость волны:                                                      &  % [12, 1]
$v = \frac{\omega}{k} = \frac{c}{n}$                                         &  % [12, 2]
                                                                             &  % [12, 3]
                                                                             \\ % [12, 4]
\hline
Длина волны:                                                                 &  % [13, 1]
$\lambda = vT = \frac{C}{nV} = \frac{2 \pi c}{n \omega} =
 \frac{\lambda_0}{n}, \lambda_0$ -- длина волны в вакууме.                   &  % [13, 2]
                                                                             &  % [13, 3]
                                                                             \\ % [13, 4]
\hline
Фаза волны:                                                                  &  % [14, 1]
$\varphi = (\vec{k}, \vec{r}) - \omega t$                                    &  % [14, 2]
                                                                             &  % [14, 3]
                                                                             \\ % [14, 4]
\hline
Связь амплитуд $\vec{H}$ и $\vec{E}$:                                        &  % [15, 1]
$\sqrt{\varepsilon} E_0 = \sqrt{\mu} H_0$                                    &  % [15, 2]
                                                                             &  % [15, 3]
                                                                             \\ % [15, 4]
\hline
Уравнения Максвелла для плоских волн:                                        &  % [16, 1]
$\begin{aligned}
\vec{k} \times \vec{E} = \frac{\omega}{c} \vec{B}, (\vec{k}, \vec{D}) = 0, \\
\vec{k} \times \vec{H} = \frac{\omega}{c} \vec{D}, (\vec{k}, \vec{B}) = 0
\end{aligned}$                                                               &  % [16, 2]
                                                                             &  % [16, 3]
                                                                             \\ % [16, 4]
\hline
Расходящаяся сферическая волна:                                              &  % [17, 1]
$A = A_0 \frac{e^{i k r-i \omega t}}{r}$                                     &  % [17, 2]
                                                                             &  % [17, 3]
                                                                             \\ % [17, 4]
\hline
Сходящаяся сферическая волна:                                                &  % [18, 1]
$A = A_0 \frac{e^{-i k r-i \omega t}}{r}$                                    &  % [18, 2]
                                                                             &  % [18, 3]
                                                                             \\ % [18, 4]
\hline
\end{tabular}

\newpage

\end{document}
