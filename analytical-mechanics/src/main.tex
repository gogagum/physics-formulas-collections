\documentclass{article}
\usepackage[utf8]{inputenc}
\usepackage[T1]{fontenc}
\usepackage[left=0cm,right=0cm,top=0cm,bottom=0cm,bindingoffset=0cm]{geometry}
\usepackage[russian]{babel}
\usepackage{amssymb,amsmath}

\begin{document}

% Page 1
\begin{tabular}{ |p{3.8cm}|p{5.7cm}|p{3.8cm}|p{5.7cm}|  }
\hline
\multicolumn{4}{|c|}{Аналитиеская механика.} \\
\hline
Положение материальной точки:                                                &  % [1, 1]
$\vec{r} = x(t) \vec{i} + y(t) \vec{j} + z(t) \vec{k}$                       &  % [1, 2]
Ускорение:                                                                   &  % [1, 3]
$\begin{aligned}
\vec{w} = w_1 \vec{e_1} + w_2 \vec{e_2} + w_3 \vec{e_3},               \\
\vec{w} = \frac{d^2 x}{d t^2} \vec{i} + \frac{d^2 y}{d t^2} \vec{j} +
\frac{d^2 z}{d t^2} \vec{k}
\end{aligned}$                                                               \\ % [1, 4]
\hline
Скорость материальной точки:                                                 &  % [2, 1]
$\vec{v} = \frac{d{\vec{r}}}{dt} =
\frac{dx}{dt} \vec{i} + \frac{dy}{dt} \vec{j} + \frac{dx}{dt} \vec{k}$       &  % [2, 2]
Производные декартовых координат через произвоные криволинейных:             &  % [2, 3]
$\begin{aligned}
\frac{d^2 \chi_i}{d t^2} = \sum_{l=1}^{l \leq 3} \sum_{m=1}^{m \leq 3}
\frac{\partial^2 \chi_i}{\partial q_l \partial q_m}
\dot{q_l} \dot{q_m} + \frac{\partial \chi_i}{\partial q_l} \ddot{q_l},  \\
\chi_0 = x, \chi_1 = y, \chi_2 = z.
\end{aligned}$                                                               \\ % [2, 4]
\hline
Ускорение материальной точки:                                                &  % [3, 1]
$\vec{w} = \frac{d{\vec{v}}}{dt} =
\frac{d^2 x}{dt^2} \vec{i} +
\frac{d^2 y}{dt^2} \vec{j} + \frac{d^2 x}{dt^2} \vec{k}$                     &  % [3, 2]
Углы Эйлера:                                                                 &  % [3, 3]
Я не очень умею техатб.                                                      \\ % [3, 4]
\hline
Вектор $\vec{\tau}$, определение:                                            &  % [4, 1]
$\begin{aligned}
\vec{v} = \frac{d}{dt} \vec{r[s(t)]} =
\frac{d\vec{r}}{ds} \frac{ds}{dt} = \vec{\tau} \frac{ds}{dt},   \\
\vec{\tau} = \frac{d\vec{r}}{ds}
\end{aligned}$                                                               &  % [4, 2]
Ортогональные отображения:                                                   &  % [4, 3]
$\begin{aligned}
\vec{R} =
\left(\begin{array}{c}
x \\
y \\
z \\
\end{array} \right),
\vec{R^{'}} =
\left(\begin{array}{c}
x^{'} \\
y^{'} \\
z^{'} \\
\end{array} \right),         \\
\vec{R^{'}} = A \vec{R}
\end{aligned}$                                                               \\ % [4, 4]
\hline
Ускорение через $\vec{\tau}$:                                                &  % [5, 1]
$\vec{w} = \frac{d\vec{v}}{dt} =
\frac{d}{dt} (v \vec{\tau}) =
\frac{dv}{dt} \vec{\tau} + v \frac{d\vec{\tau}}{ds} \frac{ds}{dt} =
\frac{dv}{dt} \vec{\tau} + v^2 \frac{d \vec{\tau}}{ds}$                      &  % [5, 2]
Ортогональное преобразование для комплексного вектора:                       &  % [5, 3]
$A (\vec{P} + i \vec{Q}) = A \vec{P} + i A \vec{Q}$                          \\ % [5, 4]
\hline
Вектор кривизны, и его связь с $\vec{n}$:                                    &  % [6, 1]
$\frac{d\vec{\tau}}{ds} = \frac{1}{\rho} \vec{n}$                            &  % [6, 2]
<<Хорошее>> определение нормы комплексного вектора:                          &  % [6, 3]
$\vec{P} + i \vec{Q} = \sqrt{(\vec{P} + i \vec{Q})(\vec{P} + i \vec{Q})} =
\sqrt{\vec{P^T}\vec{P} + \vec{Q^T}\vec{Q}}$                                  \\ % [6, 4]
\hline
Разложение $\vec{w}$ по $\vec{\tau}$ и $\vec{n}$:                            &  % [7, 1]
$\vec{w} = \frac{dv}{dt} \vec{\tau} + \frac{v^2}{\rho} \vec{n}$              &  % [7, 2]
Кватернион:                                                                  &  % [7, 3]
$\Lambda = \lambda_0 i_0 + \lambda_1 i_1 + \lambda_2 i_2 + \lambda_3 i_3$    \\ % [7, 4]
\hline
Вектор бинормали $\vec{b}$ :                                                 &  % [8, 1]
$\vec{b} = \vec{\tau} \times \vec{n}$                                        &  % [8, 2]
Свойства кватернионов:                                                       &  % [8, 3]
$\begin{aligned}
(\Lambda \circ \mathcal{M}) \circ \mathcal{N} =
 \Lambda \circ (\mathcal{M} \circ \mathcal{N}),             \\
(\Lambda+\mathcal{M}) \circ(\mathcal{N}+\mathcal{R}) =      \\
\Lambda \circ \mathcal{N} + \mathcal{M} \circ \mathcal{N} + \\
\Lambda \circ \mathcal{R}  +\mathcal{M} \circ \mathcal{R},  \\
(\lambda \Lambda) \circ(\mu \mathcal{M}) =
\lambda \mu \Lambda \circ \mathcal{M}
\end{aligned}$                                                               \\ % [8, 4]
\hline
Касательные к координатныйм линиям ($\vec{r} = \vec{r} (q_1, q_2, q_3)$):    &  % [9, 1]
$\begin{aligned}
\frac{\partial \vec{r}}{\partial q_m} =
\frac{\partial x}{\partial q_m} \vec{i} +
\frac{\partial y}{\partial q_m} \vec{j} +
\frac{\partial z}{\partial q_m} \vec{k} =  \\
H_1 \vec{e_m},
 m = 1, 2, 3
\end{aligned}$                                                               &  % [9, 2]
Умножение кватернионов:                                                      &  % [9, 3]
$\begin{aligned}
i_{0} \circ i_{k} = i_k \circ i_0 = i_k, k=0,1,2,3        \\
i_{k} \circ i_{k} = -i_{0},  k=1,2,3                      \\
i_{1} \circ i_{2} =  i_{3}, i_{2} \circ i_{3} =  i_{1},
i_{3} \circ i_{1} =  i_{2},                               \\
i_{2} \circ i_{1} = -i_{3}, i_{3} \circ i_{2} = -i_{1},   \\
i_{1} \circ i_{3} = -i_{2}                                \\
\end{aligned}$                                                               \\ % [9, 4]
\hline
Коэффициенты Ляме:                                                           &  % [10, 1]
$H_k = \sqrt{(\frac{\partial x}{\partial q_k})^2 +
 (\frac{\partial y}{\partial q_k})^2 + (\frac{\partial z}{\partial q_k})^2}$ &  % [10, 2]
Ещё свойства умножения кватернионов:                                         &  % [10, 3]
$i_0 \circ i_0 = i_0, i_0 \circ i_1 = i_1 \circ i_0 = i_1,
 i_1 \circ i_1 = -i_0$                                                       \\ % [10, 4]
\hline
Ортогональные криволинейные координаты:                                      &  % [11, 1]
$(\vec{e_1} \cdot \vec{e_2}) = (\vec{e_2} \cdot \vec{e_3}) =
 (\vec{e_3} \cdot \vec{e_1}) = 0 $                                           &  % [11, 2]
Ортогональная матрица:                                                       &  % [11, 3]
$\begin{aligned}
\vec{R^{'}} =
\left(\begin{array}{ccc}
a_{11} & a_{12} & a_{13} \\
a_{21} & a_{22} & a_{23} \\
a_{31} & a_{32} & a_{33}
\end{array} \right) =                     \\
\left(\begin{array}{ccc}
\vec{r_{1}} & \vec{r_{2}} & \vec{r_{3}}
\end{array} \right),                      \\
|\vec{r_i}| = 1,
\vec{r_i} \cdot \vec{r_j} = 0, i \neq j
\end{aligned}$                                                               \\ % [11, 4]
\hline.
Эквивалентные условия ортогональности криволинейных координат:               &  % [12, 1]
$\frac{\partial x}{\partial q_l} \frac{\partial x}{\partial q_m} =
 \frac{\partial y}{\partial q_l} \frac{\partial y}{\partial q_m} =
 \frac{\partial z}{\partial q_l} \frac{\partial z}{\partial q_m} = 0$
для $l \neq m$                                                               &  % [12, 2]
Матрица поворота относительно оси $X$ на угол $\phi$:                        &  % [12, 3]
$A = \left(
\begin{array}{ccc}
1 & 0          & 0            \\
0 & \cos(\phi) & -\sin(\phi)  \\
0 & \sin(\phi) & \cos(\phi)
\end{array}
\right)$                                                                     \\ % [12, 4]
\hline
Дифференциал дуги произвольной кривой (метрика пространства):                &  % [13, 1]
$ds^2 = dx^2 + dy^2 + dz^2 =
\sum_{i = 1}^3 \sum_{j = 1}^3 g_{ij} dq_i dq_j$                              &  % [13, 2]

Числовая интерпритация кватернионов:                                         &  % [13, 3]
$i_0$ -- $1$, $i_1$ -- $\sqrt{-1}$, то векторное пространство
$\Lambda = \lambda_0 + \lambda_1 \sqrt{-1}$ подчиняется вышеуказанным
правилам                                                                     \\ % [13, 4]
\hline
Метрика пространства (случай ортогональных координат):                       &  % [14, 1]
$ds^2 = H_1^2 d q_1^2 + H_2^2 d q_2^2 + H_3^2 d q_3^2$                       &  % [14, 2]
Матричная интерпритация кватернионов:                                        &  % [14, 3]
$\begin{aligned}
i_0 =
\left(\begin{array}{cc}
1 & 0 \\
0 & 1
\end{array}\right),
i_1 =
\left(\begin{array}{cc}
0 & -1 \\
1 & 0
\end{array}\right),      \\
\Lambda = \lambda_0 i_0 + \lambda_1 i_1
\end{aligned}$
                               \\ % [14, 4]
\hline
Скорость через криволинейные координаты:                                     &  % [15, 1]
$\vec{v} = \frac{d\vec{r}}{dt} =
 \frac{\partial \vec{r}}{\partial q_1} \dot{q_1} +
 \frac{\partial \vec{r}}{\partial q_2} \dot{q_2} +
 \frac{\partial \vec{r}}{\partial q_3} \dot{q_3} =
H_1 \dot q_1 \vec{e_1} + H_2 \dot q_2 \vec{e_2} + H_2 \dot q_2 \vec{e_2}$    &  % [15, 2]
Геометро-числовая интерпритация кватернионов:                                &  % [15, 3]
$\begin{aligned}
\Lambda = \lambda_0 + \lambda_1 \vec{i_1} +
 \lambda_2 \vec{i_2} + \lambda_2 \vec{i_2},         \\
\lambda_0 \in \mathbb{R}, \vec{\lambda} \in \mathbb{E}^3
\end{aligned}$                                                                \\ % [15, 4]
\hline
\end{tabular}

\newpage
% Page 2

\begin{tabular}{ |p{3.8cm}|p{5.7cm}|p{3.8cm}|p{5.7cm}|  }
\hline
\multicolumn{4}{|c|}{Аналитиеская механика.} \\
\hline
Произведение трехмерных ортов в геометро-числовой интерпритации:             &  % [1, 1]
$\begin{aligned}
i_k \circ i_k = -1,                      \\
i_k \circ i_l = \vec{i_k} \times \vec{i_l} (k \neq l)
\end{aligned}$                                                               &  % [1, 2]
Кватернионное сложение поворотов (активная точка зрения):                    &  % [1, 3]
$\begin{aligned}
R^\prime = \Lambda \circ R \circ \overline{\Lambda},
R^{\prime\prime} = \mathcal{M} \circ R \circ \overline{\mathcal{M}},  \\
R^{\prime\prime} =
 \mathcal{M} \circ \Lambda \circ R \circ
 \overline{\Lambda} \circ \overline{\mathcal{M}} =                    \\
 \mathcal{M} \circ \Lambda \circ R \circ
 \overline{\mathcal{M} \circ \Lambda}
\end{aligned}$                                                               \\ % [1, 4]
\hline
Произведение кватернионов геометро-числовой интерпритации:                   &  % [2, 1]
$\begin{aligned}
\Lambda = \lambda_0 + \vec{\lambda},              \\
\mathcal{\mathcal{M}} = \mu_0 + \vec{\mu},        \\
\Lambda \circ \mathcal{M} =
\lambda_0 \mu_0 - \vec{\lambda} \cdot \vec{\mu} + \\
\lambda_0 \vec{\mu} +
\mu_0 \vec{\lambda} + \vec{\lambda} \times \vec{\mu}
\end{aligned}$                                                               &  % [2, 2]
Кватернионное сложение поворотов (пассивная точка зрения)                    &  % [2, 3]
$\begin{aligned}
i_k^\prime =\Lambda \circ i_k \overline{\Lambda},    \\
R = x\vec{i_1} + y\vec{i_2} + z\vec{i_3},            \\
R^{\left(\prime\right)} =
 \overline{\Lambda} \circ
 \left( x \vec{i_1^\prime} +
 y \vec{i_2^\prime} + z \vec{i_3^\prime} \right)
 \circ \Lambda =                                     \\
 x^\prime \vec{i_1^\prime} +
 y^\prime \vec{i_2^\prime} +
 z^\prime \vec{i_3^\prime},                          \\
 \mathcal{N}= \Lambda \circ \mathcal{M}
\end{aligned}$                                   \\ % [2, 4]
\hline
Сопряженный кватернион:                                                      &  % [3, 1]
$\Lambda = \lambda_0 + \vec{\lambda},
 \overline{\Lambda} = \lambda_0 + \vec{\lambda}$                             &  % [3, 2]
Определение угловой скорости:                                                &  % [3, 3]
\[\vec{\omega} =
 \lim_{\triangle t \to 0}
 {\frac{\triangle \phi (t + \triangle t)}{\triangle t}}
 \vec{\epsilon}(t + \triangle t)\]                               \\ % [3, 4]
\hline
Норма кватерниона:                                                           &  % [4, 1]
$||\Lambda|| = \Lambda \circ \overline{\Lambda} =
 \overline{\Lambda} \circ \Lambda =
 \lambda_{0}^2 + \lambda_{1}^2 + \lambda_{2}^2 + \lambda_{3}^2$              &  % [4, 2]
Угловая скорость через кватернион:                                           &  % [4, 3]
$\begin{aligned}
\Lambda(t) =
 \cos{\frac{\phi(t + \triangle t)}{2}} +                  \\
 \vec{\epsilon}(t) \sin{\frac{\phi(t + \triangle t)}{2}}, \\
\dot{\Lambda} = \frac{1}{2} \vec{\omega} \circ \Lambda(t),
\vec{\omega} = 2 \overline{\Lambda} \circ \dot{\Lambda}
\end{aligned}$                                                               \\ % [4, 4]
\hline
Свойства (1) и (2) произведения кватернионов:                                &  % [5, 1]
$\begin{aligned}
\Lambda \circ \mathcal{M} \neq \mathcal{M} \circ \Lambda,      \\
\overline{\Lambda \circ \mathcal{M}} =
\overline{\mathcal{M}} \circ \overline{\Lambda}
\end{aligned}$                                                               &  % [5, 2]
Сложение угловых скоростей:                                                  &  % [5, 3]
$\vec{\omega} = \vec{\omega_1} + \vec{\omega_2}$ для
 $\vec{\omega_1}$ и $\vec{\omega_2}, заданных в одних и тех же осях$         \\ % [5, 4]
\hline
Свойство (3) произведения кватернионов:                                      &  % [6, 1]
$||\Lambda \circ \mathcal{M}|| =
 (\Lambda \circ \mathcal{M}) \circ \overline{(\Lambda \circ \mathcal{M})} =
 \Lambda \circ \mathcal{M} \circ \overline{\Lambda} \circ
 \overline{\mathcal{M}} = ||\Lambda|| \cdot ||\mathcal{M}||$                 &  % [6, 2]
Формула Эйлера.                                                              &  % [6, 3]
$\begin{aligned}
\vec{r^{\prime}}(t) = \Lambda(t) \circ \vec{r} \circ \overline{\Lambda}(t), \\
\dot{\vec{r}} = \vec{\omega} \times {\vec{r}}
\end{aligned}$                                                               \\ % [6, 4]
\hline
Свойство (4) произведения кватернионов:                                      &  % [7, 1]
Инвариантно относительно ортогональных преобразований в векторной
 части кватернионов.                                                         &  % [7, 2]
Формула Эйлера в матричной форме:                                            &  % [7, 3]
$\begin{aligned}
\dot{\vec{r^\prime}} = \Omega \vec{r^\prime}, \\
\Omega =
\left(\begin{array}{ccc}
 0        & -\omega_z &  \omega_y \\
 \omega_z &  0        & -\omega_x \\
-\omega_y &  \omega_x &  0
\end{array}
\right)\end{aligned}$                               \\ % [7, 4]
\hline
Свойство (5) произведения кватернионов (обратный):                           &  % [8, 1]
$\Lambda^{-1} = \frac{\overline{\Lambda}}{||\Lambda||}$                      &  % [8, 2]
Уравнение Пуассона (x, y, z).                                                &  % [8, 3]
$\begin{aligned}
\vec{r^\prime} = A(t) \vec{r},              \\
\dot{A} = \Omega A.
\end{aligned}$                                                     \\ % [8, 4]
\hline
Присоединённое отображение:                                                  &  % [9, 1]
$\mathcal{R} \rightarrow \mathcal{R}^{\prime}:
\mathcal{R}^{\prime} = Ad \mathcal{R} =
\Lambda \circ \mathcal{R} \circ \Lambda^{\prime}$                            &  % [9, 2]
Уравнение Пуассона $(\xi, \eta, \zeta)$.                                     &  % [9, 3]
$\begin{aligned}
\vec{r^\prime} = A(t) \vec{r},              \\
\dot{A} = A \Omega
\end{aligned}$                                                               \\ % [9, 4]
\hline
Свойства (1) и (2) присоединенного отображения:                              &  % [10, 1]
Не меняет скалярной части,
 действует на векторную часть как линейное преобразование.                   &  % [10, 2]
Уравнение Пуассона в кватернионах $(x, y, z)$:                               &  % [10, 3]
$2\dot{\Lambda} = \vec{\omega} \circ \Lambda$                                \\ % [10, 4]
\hline
Поворот через кватернион:                                                    &  % [11, 1]
$\Lambda = \lambda_0 + \lambda \vec{e},
\Lambda = \cos{\frac{\phi}{2}} + \vec{e} \sin{\frac{\phi}{2}}$ ---
поворот на угол $\phi$ вокруг $\vec{e}$.                                     &  % [11, 2]
Уравнение Пуассона в кватернионах $(\xi, \eta, \zeta)$:                      &  % [11, 3]
$2\dot{\Lambda} = \Lambda \circ \vec{\omega}$                                \\ % [11, 4]
\hline
Группа $SO(3)$:                                                              &  % [12, 1]
$R^\prime = A R, R^{\prime\prime} = B R^\prime, R^{\prime\prime} = (BA) R,
C = BA$---матрица, задающая суммарный поворот.                               &  % [12, 2]
Относительное движениие. Обозначения.                                        &  % [12, 3]
$\begin{aligned}
\vec{r} =
\left(\begin{array}{c}
\xi  \\
\eta \\
\zeta
\end{array}\right),
\vec{r_0} =
\left(\begin{array}{c}
x \\
y \\
z
\end{array}\right),                           \\
\vec{v_{rel}} = \dot{\vec{r}},
\vec{w_{rel}} = \ddot{\vec{r}},
\vec{v_0} = \dot{\vec{r_0}},
\vec{w_0} = \ddot{\vec{r_0}}
\end{aligned}$                                                               \\ % [12, 4]
\hline
Активная точка зрения.                                                       &  % [13, 1]
Матрицы последовательных поворотов перемножаются в обратном порядке.
Все матрицы вычисляются в общем для всех базисе $\vec{i}, \vec{j}, \vec{k}$. &  % [13, 2]
Сложение скоростей:                                                          &  % [13, 3]
$\vec{v} = \vec{v_0} + \vec{\omega} \times \vec{r} + \vec{v_{rel}}$          \\ % [13, 4]
\hline
Пассивная точка зрения.                                                      &  % [14, 1]
Матрицы последовательных поворотов перемножаются в прямом порядке.
Каждая матрица рассматривается в поворачиваемом ею базисе.                   &  % [14, 2]
Сложение ускорений:                                                          &  % [14, 3]
$\vec{w} = \vec{w_0} + \vec{\omega} \times (\vec{\omega} \times \vec{r}) +
 \dot{\vec{\omega}} \times \vec{r} + 2 \vec{\omega} \times \vec{v_{rel}} +
 \vec{w_{rel}}$                           \\ % [14, 4]
\hline
Ещё про активную точку зрения (сложение поворотов):                          &  % [15, 1]
$R^{\left(\prime\right)} =
\left(\begin{array}{c}
x^\prime \\
y^\prime \\
z^\prime \\
\end{array}\right) =
A^{T}
\left(\begin{array}{c}
x \\
y \\
z \\
\end{array}\right)$                                                          &  % [15, 2]
                                                                             &  % [15, 3]
                                                                             \\ % [15, 4]
\hline
\end{tabular}

\newpage
% page 3
\begin{tabular}{ |p{3.8cm}|p{5.7cm}|p{6cm}|p{3.5cm}|  }
\hline
\multicolumn{4}{|c|}{Аналитиеская механика.} \\
\hline
Конфигурационное многообразие и число степеней свободы:                      &  % [1, 1]
$M \subset \mathbb{R}^m$, однозначно отображаемое в множество возможных
 положений системы. $m$ - число степеней свободы.                            &  % [1, 2]
Обобщённый интеграл энергии:                                                 &  % [1, 3]
$\sum_i {\dot{q_i} \frac{\partial \mathcal{L}}{\partial \dot{q_i}}} -
 \mathcal{L} = const$                                       \\ % [1, 4]
\hline
Параметризация и лагранжевы параметры.                                       &  % [2, 1]
$R = R(\nu, t, q_1, \dots, q_n), q_1, \dots, q_n$ --- лагранжевы параметры.  &  % [2, 2]
                                                                             &  % [2, 3]
                                                                             \\ % [2, 4]
\hline
Стационарная параметризация:                                                 &  % [3, 1]
Параметризация не зависящая явно от времени.                                 &  % [3, 2]
                                                                             &  % [3, 3]
                                                                             \\ % [3, 4]
\hline
Нестационарная параметризация:                                               &  % [4, 1]
Параметризация явно зависящая от времени.                                    &  % [4, 2]
                                                                             &  % [4, 3]
                                                                             \\ % [4, 4]
\hline
Кинематически независимые координаты и голономные системы.                   &  % [5, 1]
Локальные координаты не стеснены никакими дополнительными условиями типа
$f_k(t, q, \dot{q}) = 0, k = 1, \dots, s$, а такие механические системы
называются голономными.                                                      &  % [5, 2]
                                                                             &  % [5, 3]
                                                                             \\ % [5, 4]
\hline
Виртуальное перемещение:                                                     &  % [6, 1]
$\delta R = \sum_i{\frac{\partial R}{\partial q_i} \delta q_i}$              &  % [6, 2]
                                                                             &  % [6, 3]
                                                                             \\ % [6, 4]
\hline
Обобщённые силы:                                                             &  % [7, 1]
$Q_i = \int \frac{\partial R}{\partial q_i} \cdot F^d dm$                    &  % [7, 2]
                                                                             &  % [7, 3]
                                                                             \\ % [7, 4]
\hline
Кинетическая энергия:                                                        &  % [8, 1]
$T = \int {\vec{V} \cdot \vec{V} dm}$                                        &  % [8, 2]
                                                                             &  % [8, 3]
                                                                             \\ % [8, 4]
\hline
Уравнение Лагранжа:                                                          &  % [9, 1]
$\frac{d}{dt} \frac{\partial T}{\partial \dot{q_i}} -
\frac{\partial T}{\partial q_i} = Q_i (i = 1, \dots, n)$                     &  % [9, 2]
                                                                             &  % [9, 3]
                                                                             \\ % [9, 4]
\hline
Потенциальная сила:                                                          &  % [10, 1]
Существует $U$, что $Q_i = \frac{\partial U}{\partial q_i} (i = 1, \dots, n)$&  % [10, 2]
                                                                             &  % [10, 3]
                                                                             \\ % [10, 4]
\hline
Обощённо потенциальная сила:                                                 &  % [11, 1]
Существует $U$, что $Q_i = \frac{\partial U}{\partial q_i} -
 \frac{d}{dt} \frac{\partial U}{\partial \dot{q_i}} (i = 1, \dots, n)$       &  % [11, 2]
                                                                             &  % [11, 3]
                                                                             \\ % [11, 4]
\hline
Функция Лагранжа:                                                            &  % [12, 1]
$\mathcal{L} = T + U$                                                        &  % [12, 2]
                                                                             &  % [12, 3]
                                                                             \\ % [12, 4]
\hline
Уравнение Лагранжа через функцию Лагранжа:                                   &  % [13, 1]
$\frac{d}{dt} \frac{\partial \mathcal{L}}{\partial \dot{q_i}} -
 \frac{\partial \mathcal{L}}{\partial q_i} = 0 (i = 1, \dots, n)$            &  % [13, 2]
                                                                             &  % [13, 3]
                                                                             \\ % [13, 4]
\hline
Ковариантность уравнений Лагранжа:                                           &  % [14, 1]
Если обощённые координаты подвергнуть преобразованиям $q_i \to \tilde{q_i}$
 из $C_2$:
\[q_i = q_i(t, \tilde{q}) \]
то в новых переменных уравнения Лагранжа сохраняют форму.                    &  % [14, 2]
                                                                             &  % [14, 3]
                                                                             \\ % [14, 4]
\hline
Невырожденность уравнений Лагранжа:                                          &  % [15, 1]
$\dots$                                                                      &  % [15, 2]
                                                                             &  % [15, 3]
                                                                             \\ % [15, 4]
\hline
\end{tabular}

\end{document}
