\documentclass{article}
\usepackage[utf8]{inputenc}
\usepackage[T1]{fontenc}
\usepackage[left=0cm,right=0cm,top=0cm,bottom=0cm,bindingoffset=0cm]{geometry}
\usepackage[russian]{babel}
\usepackage{amssymb,amsmath}

\begin{document}

% Page 1
\begin{tabular}{ |p{3.8cm}|p{5.7cm}|p{3.8cm}|p{5.7cm}|  }
\hline
\multicolumn{4}{|c|}{Аналитиеская механика.} \\
\hline
Положение материальной точки:                                                &  % [1, 1]
$\vec{r} = x(t) \vec{i} + y(t) \vec{j} + z(t) \vec{k}$                       &  % [1, 2]
Ускорение:                                                                   &  % [1, 3]
$\begin{aligned}
\vec{w} = w_1 \vec{e_1} + w_2 \vec{e_2} + w_3 \vec{e_3},               \\
\vec{w} = \frac{d^2 x}{d t^2} \vec{i} + \frac{d^2 y}{d t^2} \vec{j} +
\frac{d^2 z}{d t^2} \vec{k}
\end{aligned}$                                                               \\ % [1, 4]
\hline
Скорость материальной точки:                                                 &  % [2, 1]
$\vec{v} = \frac{d{\vec{r}}}{dt} =
\frac{dx}{dt} \vec{i} + \frac{dy}{dt} \vec{j} + \frac{dx}{dt} \vec{k}$       &  % [2, 2]
Производные декартовых координат через произвоные криволинейных:             &  % [2, 3]
$\begin{aligned}
\frac{d^2 \chi_i}{d t^2} = \sum_{l=1}^{l \leq 3} \sum_{m=1}^{m \leq 3}
\frac{\partial^2 \chi_i}{\partial q_l \partial q_m}
\dot{q_l} \dot{q_m} + \frac{\partial \chi_i}{\partial q_l} \ddot{q_l},  \\
\chi_0 = x, \chi_1 = y, \chi_2 = z.
\end{aligned}$                                                               \\ % [2, 4]
\hline
Ускорение материальной точки:                                                &  % [3, 1]
$\vec{w} = \frac{d{\vec{v}}}{dt} =
\frac{d^2 x}{dt^2} \vec{i} +
\frac{d^2 y}{dt^2} \vec{j} + \frac{d^2 x}{dt^2} \vec{k}$                     &  % [3, 2]
Углы Эйлера:                                                                 &  % [3, 3]
Я не очень умею техатб.                                                      \\ % [3, 4]
\hline
Вектор $\vec{\tau}$, определение:                                            &  % [4, 1]
$\begin{aligned}
\vec{v} = \frac{d}{dt} \vec{r[s(t)]} =
\frac{d\vec{r}}{ds} \frac{ds}{dt} = \vec{\tau} \frac{ds}{dt},   \\
\vec{\tau} = \frac{d\vec{r}}{ds}
\end{aligned}$                                                               &  % [4, 2]
Ортогональные отображения:                                                   &  % [4, 3]
$\begin{aligned}
\vec{R} =
\left(\begin{array}{c}
x \\
y \\
z \\
\end{array} \right),
\vec{R^{'}} =
\left(\begin{array}{c}
x^{'} \\
y^{'} \\
z^{'} \\
\end{array} \right),         \\
\vec{R^{'}} = A \vec{R}
\end{aligned}$                                                               \\ % [4, 4]
\hline
Ускорение через $\vec{\tau}$:                                                &  % [5, 1]
$\vec{w} = \frac{d\vec{v}}{dt} =
\frac{d}{dt} (v \vec{\tau}) =
\frac{dv}{dt} \vec{\tau} + v \frac{d\vec{\tau}}{ds} \frac{ds}{dt} =
\frac{dv}{dt} \vec{\tau} + v^2 \frac{d \vec{\tau}}{ds}$                      &  % [5, 2]
Ортогональное преобразование для комплексного вектора:                       &  % [5, 3]
$A (\vec{P} + i \vec{Q}) = A \vec{P} + i A \vec{Q}$                          \\ % [5, 4]
\hline
Вектор кривизны, и его связь с $\vec{n}$:                                    &  % [6, 1]
$\frac{d\vec{\tau}}{ds} = \frac{1}{\rho} \vec{n}$                            &  % [6, 2]
<<Хорошее>> определение нормы комплексного вектора:                          &  % [6, 3]
$\vec{P} + i \vec{Q} = \sqrt{(\vec{P} + i \vec{Q})(\vec{P} + i \vec{Q})} =
\sqrt{\vec{P^T}\vec{P} + \vec{Q^T}\vec{Q}}$                                  \\ % [6, 4]
\hline
Разложение $\vec{w}$ по $\vec{\tau}$ и $\vec{n}$:                            &  % [7, 1]
$\vec{w} = \frac{dv}{dt} \vec{\tau} + \frac{v^2}{\rho} \vec{n}$              &  % [7, 2]
Кватернион:                                                                  &  % [7, 3]
$\Lambda = \lambda_0 i_0 + \lambda_1 i_1 + \lambda_2 i_2 + \lambda_3 i_3$    \\ % [7, 4]
\hline
Вектор бинормали $\vec{b}$ :                                                 &  % [8, 1]
$\vec{b} = \vec{\tau} \times \vec{n}$                                        &  % [8, 2]
Свойства кватернионов:                                                       &  % [8, 3]
$\begin{aligned}
(\Lambda \circ \mathcal{M}) \circ \mathcal{N} =
 \Lambda \circ (\mathcal{M} \circ \mathcal{N}),             \\
(\Lambda+\mathcal{M}) \circ(\mathcal{N}+\mathcal{R}) =      \\
\Lambda \circ \mathcal{N} + \mathcal{M} \circ \mathcal{N} + \\
\Lambda \circ \mathcal{R}  +\mathcal{M} \circ \mathcal{R},  \\
(\lambda \Lambda) \circ(\mu \mathcal{M}) =
\lambda \mu \Lambda \circ \mathcal{M}
\end{aligned}$                                                               \\ % [8, 4]
\hline
Касательные к координатныйм линиям ($\vec{r} = \vec{r} (q_1, q_2, q_3)$):    &  % [9, 1]
$\begin{aligned}
\frac{\partial \vec{r}}{\partial q_m} =
\frac{\partial x}{\partial q_m} \vec{i} +
\frac{\partial y}{\partial q_m} \vec{j} +
\frac{\partial z}{\partial q_m} \vec{k} =  \\
H_1 \vec{e_m},
 m = 1, 2, 3
\end{aligned}$                                                               &  % [9, 2]
Умножение кватернионов:                                                      &  % [9, 3]
$\begin{aligned}
i_{0} \circ i_{k} = i_k \circ i_0 = i_k, k=0,1,2,3        \\
i_{k} \circ i_{k} = -i_{0},  k=1,2,3                      \\
i_{1} \circ i_{2} =  i_{3}, i_{2} \circ i_{3} =  i_{1},
i_{3} \circ i_{1} =  i_{2},                               \\
i_{2} \circ i_{1} = -i_{3}, i_{3} \circ i_{2} = -i_{1},   \\
i_{1} \circ i_{3} = -i_{2}                                \\
\end{aligned}$                                                               \\ % [9, 4]
\hline
Коэффициенты Ляме:                                                           &  % [10, 1]
$H_k = \sqrt{(\frac{\partial x}{\partial q_k})^2 +
 (\frac{\partial y}{\partial q_k})^2 + (\frac{\partial z}{\partial q_k})^2}$ &  % [10, 2]
Ещё свойства умножения кватернионов:                                         &  % [10, 3]
$i_0 \circ i_0 = i_0, i_0 \circ i_1 = i_1 \circ i_0 = i_1,
 i_1 \circ i_1 = -i_0$                                                       \\ % [10, 4]
\hline
Ортогональные криволинейные координаты:                                      &  % [11, 1]
$(\vec{e_1} \cdot \vec{e_2}) = (\vec{e_2} \cdot \vec{e_3}) =
 (\vec{e_3} \cdot \vec{e_1}) = 0 $                                           &  % [11, 2]
Ортогональная матрица:                                                       &  % [11, 3]
$\begin{aligned}
\vec{R^{'}} =
\left(\begin{array}{ccc}
a_{11} & a_{12} & a_{13} \\
a_{21} & a_{22} & a_{23} \\
a_{31} & a_{32} & a_{33}
\end{array} \right) =                     \\
\left(\begin{array}{ccc}
\vec{r_{1}} & \vec{r_{2}} & \vec{r_{3}}
\end{array} \right),                      \\
|\vec{r_i}| = 1,
\vec{r_i} \cdot \vec{r_j} = 0, i \neq j
\end{aligned}$                                                               \\ % [11, 4]
\hline.
Эквивалентные условия ортогональности криволинейных координат:               &  % [12, 1]
$\frac{\partial x}{\partial q_l} \frac{\partial x}{\partial q_m} =
 \frac{\partial y}{\partial q_l} \frac{\partial y}{\partial q_m} =
 \frac{\partial z}{\partial q_l} \frac{\partial z}{\partial q_m} = 0$
для $l \neq m$                                                               &  % [12, 2]
Матрица поворота относительно оси $X$ на угол $\phi$:                        &  % [12, 3]
$A = \left(
\begin{array}{ccc}
1 & 0          & 0            \\
0 & \cos(\phi) & -\sin(\phi)  \\
0 & \sin(\phi) & \cos(\phi)
\end{array}
\right)$                                                                     \\ % [12, 4]
\hline
Дифференциал дуги произвольной кривой (метрика пространства):                &  % [13, 1]
$ds^2 = dx^2 + dy^2 + dz^2 =
\sum_{i = 1}^3 \sum_{j = 1}^3 g_{ij} dq_i dq_j$                              &  % [13, 2]

Числовая интерпритация кватернионов:                                         &  % [13, 3]
$i_0$ -- $1$, $i_1$ -- $\sqrt{-1}$, то векторное пространство
$\Lambda = \lambda_0 + \lambda_1 \sqrt{-1}$ подчиняется вышеуказанным
правилам                                                                     \\ % [13, 4]
\hline
Метрика пространства (случай ортогональных координат):                       &  % [14, 1]
$ds^2 = H_1^2 d q_1^2 + H_2^2 d q_2^2 + H_3^2 d q_3^2$                       &  % [14, 2]
Матричная интерпритация кватернионов:                                        &  % [14, 3]
$\begin{aligned}
i_0 =
\left(\begin{array}{cc}
1 & 0 \\
0 & 1
\end{array}\right),
i_1 =
\left(\begin{array}{cc}
0 & -1 \\
1 & 0
\end{array}\right),      \\
\Lambda = \lambda_0 i_0 + \lambda_1 i_1
\end{aligned}$
                               \\ % [14, 4]
\hline
Скорость через криволинейные координаты:                                     &  % [15, 1]
$\vec{v} = \frac{d\vec{r}}{dt} =
 \frac{\partial \vec{r}}{\partial q_1} \dot{q_1} +
 \frac{\partial \vec{r}}{\partial q_2} \dot{q_2} +
 \frac{\partial \vec{r}}{\partial q_3} \dot{q_3} =
H_1 \dot q_1 \vec{e_1} + H_2 \dot q_2 \vec{e_2} + H_2 \dot q_2 \vec{e_2}$    &  % [15, 2]
Геометро-числовая интерпритация кватернионов:                                &  % [15, 3]
$\begin{aligned}
\Lambda = \lambda_0 + \lambda_1 \vec{i_1} +
 \lambda_2 \vec{i_2} + \lambda_2 \vec{i_2},         \\
\lambda_0 \in \mathbb{R}, \vec{\lambda} \in \mathbb{E}^3
\end{aligned}$                                                                \\ % [15, 4]
\hline
\end{tabular}

\newpage
% Page 2

\begin{tabular}{ |p{3.8cm}|p{5.7cm}|p{6cm}|p{3.5cm}|  }
\hline
\multicolumn{4}{|c|}{Аналитиеская механика.} \\
\hline
Произведение трехмерных ортов в геометро-числовой интерпритации:             &  % [1, 1]
$\begin{aligned}
i_k \circ i_k = -1,                      \\
i_k \circ i_l = \vec{i_k} \times \vec{i_l} (k \neq l)
\end{aligned}$                                                               &  % [1, 2]
                                                                             &  % [1, 3]
                                                                             \\ % [1, 4]
\hline
Произведение кватернионов геометро-числовой интерпритации:                   &  % [2, 1]
$\begin{aligned}
\Lambda = \lambda_0 + \vec{\lambda},              \\
M = \mu_0 + \vec{\mu},                            \\
\Lambda \circ M =
\lambda_0 \mu_0 - \vec{\lambda} \cdot \vec{\mu} + \\
\lambda_0 \vec{\mu} +
\mu_0 \vec{\lambda} + \vec{\lambda} \times \vec{\mu}
\end{aligned}$                                                               &  % [2, 2]
                                                                             &  % [2, 3]
                                                                             \\ % [2, 4]
\hline
Сопряженный кватернион:                                                      &  % [3, 1]
$\Lambda = \lambda_0 + \vec{\lambda},
 \overline{\Lambda} = \lambda_0 + \vec{\lambda}$                             &  % [3, 2]
                                                                             &  % [3, 3]
                                                                             \\ % [3, 4]
\hline
Норма кватерниона:                                                           &  % [4, 1]
$||\Lambda|| = \Lambda \circ \overline{\Lambda} =
 \overline{\Lambda} \circ \Lambda =
 \lambda_{0}^2 + \lambda_{1}^2 + \lambda_{2}^2 + \lambda_{3}^2$              &  % [4, 2]
                                                                             &  % [4, 3]
                                                                             \\ % [4, 4]
\hline
Свойства (1) и (2) произведения кватернионов:                                &  % [5, 1]
$\begin{aligned}
\Lambda \circ M \neq M \circ \Lambda,      \\
\overline{\Lambda \circ M} = \overline{\Lambda} \circ \overline{M}
\end{aligned}$                                                               &  % [5, 2]
                                                                             &  % [5, 3]
                                                                             \\ % [5, 4]
\hline
Свойство (3) произведения кватернионов:                                      &  % [6, 1]
$||\Lambda \circ M|| =
 (\Lambda \circ M) \circ \overline{(\Lambda \circ M)} =
 \Lambda \circ M \circ \overline{\Lambda} \circ \overline{M} =
 ||\Lambda|| \cdot ||M||$                                                    &  % [6, 2]
                                                                             &  % [6, 3]
                                                                             \\ % [6, 4]
\hline
Свойство (4) произведения кватернионов:                                      &  % [7, 1]
Инвариантно относительно ортогональных преобразований в векторной
 части кватернионов.                                                         &  % [7, 2]
                                                                             &  % [7, 3]
                                                                             \\ % [7, 4]
\hline
Свойство (5) произведения кватернионов (обратный):                           &  % [8, 1]
$\Lambda^{-1} = \frac{\overline{\Lambda}}{||\Lambda||}$                      &  % [8, 2]
                                                                             &  % [8, 3]
                                                                             \\ % [8, 4]
\hline
Присоединённое отображение:                                                  &  % [9, 1]
$\mathcal{R} \rightarrow \mathcal{R}^{\prime}:
\mathcal{R}^{\prime} = Ad \mathcal{R} =
\Lambda \circ \mathcal{R} \circ \Lambda^{\prime}$                            &  % [9, 2]
                                                                             &  % [9, 3]
                                                                             \\ % [9, 4]
\hline
                                                                             &  % [10, 1]
                                                                             &  % [10, 2]
                                                                             &  % [10, 3]
                                                                             \\ % [10, 4]
\hline
                                                                             &  % [11, 1]
                                                                             &  % [11, 2]
                                                                             &  % [11, 3]
                                                                             \\ % [11, 4]
\hline
                                                                             &  % [12, 1]
                                                                             &  % [12, 2]
                                                                             &  % [12, 3]
                                                                             \\ % [12, 4]
\hline
                                                                             &  % [13, 1]
                                                                             &  % [13, 2]
                                                                             &  % [13, 3]
                                                                             \\ % [13, 4]
\hline
                                                                             &  % [14, 1]
                                                                             &  % [14, 2]
                                                                             &  % [14, 3]
                                                                             \\ % [14, 4]
\hline
                                                                             &  % [15, 1]
                                                                             &  % [15, 2]
                                                                             &  % [15, 3]
                                                                             \\ % [15, 4]
\hline
                                                                             &  % [16, 1]
                                                                             &  % [16, 2]
                                                                             &  % [16, 3]
                                                                             \\ % [16, 4]
\hline
                                                                             &  % [17, 1]
                                                                             &  % [17, 2]
                                                                             &  % [17, 3]
                                                                             \\ % [17, 4]
\hline
                                                                             &  % [18, 1]
                                                                             &  % [18, 2]
                                                                             &  % [18, 3]
                                                                             \\ % [18, 4]
\hline
                                                                             &  % [19, 1]
                                                                             &  % [19, 2]
                                                                             &  % [19, 3]
                                                                             \\ % [19, 4]
\hline
                                                                             &  % [20, 1]
                                                                             &  % [20, 2]
                                                                             &  % [20, 3]
                                                                             \\ % [20, 4]
\hline
                                                                             &  % [21, 1]
                                                                             &  % [21, 2]
                                                                             &  % [21, 3]
                                                                             \\ % [21, 4]
\hline
                                                                             &  % [22, 1]
                                                                             &  % [22, 2]
                                                                             &  % [22, 3]
                                                                             \\ % [22, 4]
\hline
                                                                             &  % [23, 1]
                                                                             &  % [23, 2]
                                                                             &  % [23, 3]
                                                                             \\ % [23, 4]
\hline
                                                                             &  % [24, 1]
                                                                             &  % [24, 2]
                                                                             &  % [24, 3]
                                                                             \\ % [24, 4]
\hline
                                                                             &  % [25, 1]
                                                                             &  % [25, 2]
                                                                             &  % [25, 3]
                                                                             \\ % [25, 4]
\hline
\end{tabular}

\end{document}
