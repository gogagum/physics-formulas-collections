\documentclass{article}
\usepackage[utf8]{inputenc}
\usepackage[T1]{fontenc}
\usepackage[left=0cm,right=0cm,top=0cm,bottom=0cm,bindingoffset=0cm]{geometry}
\usepackage[russian]{babel}
\usepackage{amssymb,amsmath}

\begin{document}
\begin{tabular}{ |p{3.8cm}|p{5.7cm}|p{3.8cm}|p{5.7cm}|  }
\hline
\multicolumn{4}{|c|}{Аналитиеская механика.} \\
\hline
Положение материальной точки:                                                &  % [1, 1]
$\vec{r} = x(t) \vec{i} + y(t) \vec{j} + z(t) \vec{k}$                       &  % [1, 2]
Углы Эйлера:                                                                 &  % [18, 1]
Я не очень умею техатб.                                                      \\ % [1, 4]
\hline
Скорость материальной точки:                                                 &  % [2, 1]
$\vec{v} = \frac{d{\vec{r}}}{dt} =
\frac{dx}{dt} \vec{i} + \frac{dy}{dt} \vec{j} + \frac{dx}{dt} \vec{k}$       &  % [2, 2]
Ортогональные отображения:                                                   &  % [1, 3]
$\begin{aligned}
\vec{R} = 
\left(\begin{array}{c}
x \\
y \\
z \\
\end{array} \right),     
\vec{R^{'}} = 
\left(\begin{array}{c}
x^{'} \\
y^{'} \\
z^{'} \\
\end{array} \right),         \\
\vec{R^{'}} = A \vec{R}
\end{aligned}$                                                               \\ % [1, 4]
\hline
Ускорение материальной точки:                                                &  % [3, 1]
$\vec{w} = \frac{d{\vec{v}}}{dt} =
\frac{d^2 x}{dt^2} \vec{i} +
\frac{d^2 y}{dt^2} \vec{j} + \frac{d^2 x}{dt^2} \vec{k}$                     &  % [3, 2]
Ортогональное преобразование для комплексного вектора:                       &  % [2, 3]
$A (\vec{P} + i \vec{Q}) = A \vec{P} + i A \vec{Q}$                          \\ % [2, 4]
\hline
Вектор $\vec{\tau}$, определение:                                            &  % [4, 1]
$\begin{aligned}
\vec{v} = \frac{d}{dt} \vec{r[s(t)]} =                         
\frac{d\vec{r}}{ds} \frac{ds}{dt} = \vec{\tau} \frac{ds}{dt},   \\
\vec{\tau} = \frac{d\vec{r}}{ds}
\end{aligned}$                                                               &  % [4, 2]
<<Хорошее>> определение нормы комплексного вектора:                          &  % [3, 3]
$\vec{P} + i \vec{Q} = \sqrt{(\vec{P} + i \vec{Q})(\vec{P} + i \vec{Q})} = 
\sqrt{\vec{P^T}\vec{P} + \vec{Q^T}\vec{Q}}$                                  \\ % [3, 4]
\hline
Ускорение через $\vec{\tau}$:                                                &  % [5, 1]
$\vec{w} = \frac{d\vec{v}}{dt} =
\frac{d}{dt} (v \vec{\tau}) =
\frac{dv}{dt} \vec{\tau} + v \frac{d\vec{\tau}}{ds} \frac{ds}{dt} =
\frac{dv}{dt} \vec{\tau} + v^2 \frac{d \vec{\tau}}{ds}$                      &  % [5, 2]
Кватернион:                                                                  &  % [4, 3]
$\Lambda = \lambda_0 i_0 + \lambda_1 i_1 + \lambda_2 i_2 + \lambda_3 i_3$    \\ % [4, 4]
\hline
Вектор кривизны, и его связь с $\vec{n}$:                                    &  % [6, 1]
$\frac{d\vec{\tau}}{ds} = \frac{1}{\rho} \vec{n}$                            &  % [6, 2]
Свойства кватернионов:                                                       &  % [5, 3]
$\begin{aligned}
(\Lambda \circ \mathcal{M}) \circ \mathcal{N} = 
 \Lambda \circ (\mathcal{M} \circ \mathcal{N}),             \\
(\Lambda+\mathcal{M}) \circ(\mathcal{N}+\mathcal{R}) =      \\
\Lambda \circ \mathcal{N} + \mathcal{M} \circ \mathcal{N} + \\
\Lambda \circ \mathcal{R}  +\mathcal{M} \circ \mathcal{R},  \\
(\lambda \Lambda) \circ(\mu \mathcal{M}) = 
\lambda \mu \Lambda \circ \mathcal{M}                              
\end{aligned}$                                                               \\ % [5, 4]
\hline
Разложение $\vec{w}$ по $\vec{\tau}$ и $\vec{n}$:                            &  % [7, 1]
$\vec{w} = \frac{dv}{dt} \vec{\tau} + \frac{v^2}{\rho} \vec{n}$              &  % [7, 2]
Умножение кватернионов:                                                      &  % [6, 3]
$\begin{aligned}
i_{0} \circ i_{k} = i_k \circ i_0 = i_k, k=0,1,2,3        \\
i_{k} \circ i_{k} = -i_{0},  k=1,2,3                      \\
i_{1} \circ i_{2} =  i_{3}, i_{2} \circ i_{3} =  i_{1}, 
i_{3} \circ i_{1} =  i_{2},                               \\
i_{2} \circ i_{1} = -i_{3}, i_{3} \circ i_{2} = -i_{1},   \\
i_{1} \circ i_{3} = -i_{2}                                \\
\end{aligned}$                                                               \\ % [6, 4]
\hline
Вектор бинормали $\vec{b}$ :                                                 &  % [8, 1]
$\vec{b} = \vec{\tau} \times \vec{n}$                                        &  % [8, 2]
                                                                             &  % [8, 3]
                                                                             \\ % [8, 4]
\hline
Касательные к координатныйм линиям ($\vec{r} = \vec{r} (q_1, q_2, q_3)$):    &  % [9, 1]
$\begin{aligned}
\frac{\partial \vec{r}}{\partial q_m} =
\frac{\partial x}{\partial q_m} \vec{i} +
\frac{\partial y}{\partial q_m} \vec{j} +
\frac{\partial z}{\partial q_m} \vec{k} =  \\
H_1 \vec{e_m}, 
 m = 1, 2, 3 
\end{aligned}$                                                               &  % [9, 2]
                                                                             &  % [9, 3]
                                                                             \\ % [9, 4]
\hline
Коэффициенты Ляме:                                                           &  % [10, 1]
$H_k = \sqrt{(\frac{\partial x}{\partial q_k})^2 +
 (\frac{\partial y}{\partial q_k})^2 + (\frac{\partial z}{\partial q_k})^2}$ &  % [10, 2]
                                                                             &  % [10, 3]
                                                                             \\ % [10, 4]
\hline
Ортогональные криволинейные координаты:                                      &  % [11, 1]
$(\vec{e_1} \cdot \vec{e_2}) = (\vec{e_2} \cdot \vec{e_3}) =
 (\vec{e_3} \cdot \vec{e_1}) = 0 $                                           &  % [11, 2]
                                                                             &  % [11, 3]
                                                                             \\ % [11, 4]
\hline
Эквивалентные условия ортогональности криволинейных координат:               &  % [12, 1]
$\frac{\partial x}{\partial q_l} \frac{\partial x}{\partial q_m} =            
 \frac{\partial y}{\partial q_l} \frac{\partial y}{\partial q_m} =
 \frac{\partial z}{\partial q_l} \frac{\partial z}{\partial q_m} = 0$
для $l \neq m$                                                               &  % [12, 2]
                                                                             &  % [12, 3]
                                                                             \\ % [12, 4]
\hline
Дифференциал дуги произвольной кривой (метрика пространства):                &  % [13, 1]
$\begin{aligned}
ds^2 = dx^2 + dy^2 + dz^2 =                       \\
\sum_{i = 1}^3 \sum_{j = 1}^3 g_{ij} dq_i dq_j
\end{aligned}$                                                               &  % [13, 2]
                                                                             &  % [13, 3]
                                                                             \\ % [13, 4]
\hline
Метрика пространства (случай ортогональных координат):                       &  % [14, 1]
$ds^2 = H_1^2 d q_1^2 + H_2^2 d q_2^2 + H_3^2 d q_3^2$                           &  % [14, 2]
                                                                             &  % [14, 3]
                                                                             \\ % [14, 4]
\hline
Скорость через криволинейные координаты:                                     &  % [15, 1]
$\vec{v} = \frac{d\vec{r}}{dt} =
 \frac{\partial \vec{r}}{\partial q_1} \dot{q_1} +
 \frac{\partial \vec{r}}{\partial q_2} \dot{q_2} +
 \frac{\partial \vec{r}}{\partial q_3} \dot{q_3} = 
H_1 \dot q_1 \vec{e_1} + H_2 \dot q_2 \vec{e_2} + H_2 \dot q_2 \vec{e_2}$    &  % [15, 2]
                                                                             &  % [15, 3]
                                                                             \\ % [15, 4]
\hline
Ускорение:                                                                   &  % [16, 1]
$\begin{aligned}
\vec{w} = w_1 \vec{e_1} + w_2 \vec{e_2} + w_3 \vec{e_3},               \\
\vec{w} = \frac{d^2 x}{d t^2} \vec{i} + \frac{d^2 y}{d t^2} \vec{j} +
\frac{d^2 z}{d t^2} \vec{k} 
\end{aligned}$                                                               &  % [16, 2]
                                                                             &  % [16, 3]
                                                                             \\ % [16, 4]
\hline
Производные декартовых координат через произвоные криволинейных:             &  % [17, 1]
$\begin{aligned}
\frac{d^2 x}{d t^2} = \sum_{l=1}^3 \sum_{m=1}^3
\frac{\partial^2 x}{\partial q_l \partial q_m}
\dot{q_l} \dot{q_m} + \frac{\partial x}{\partial q_l} \ddot{q_l},  \\
\frac{d^2 y}{d t^2} = \sum_{l=1}^3 \sum_{m=1}^3
\frac{\partial^2 y}{\partial q_l \partial q_m}
\dot{q_l} \dot{q_m} + \frac{\partial y}{\partial q_l} \ddot{q_l},  \\
\frac{d^2 z}{d t^2} = \sum_{l=1}^3 \sum_{m=1}^3
\frac{\partial^2 z}{\partial q_l \partial q_m}
\dot{q_l} \dot{q_m} + \frac{\partial z}{\partial q_l} \ddot{q_l}
\end{aligned}$                                                               &  % [17, 2]
                                                                             &  % [17, 3]
                                                                             \\ % [17, 4]
\hline
\end{tabular}

\newpage

\end{document}