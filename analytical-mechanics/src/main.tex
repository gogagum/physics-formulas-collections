\documentclass{article}
\usepackage[utf8]{inputenc}
\usepackage[T1]{fontenc}
\usepackage[left=0cm,right=0cm,top=0cm,bottom=0cm,bindingoffset=0cm]{geometry}
\usepackage[russian]{babel}
\usepackage{amssymb,amsmath}

\begin{document}
\begin{tabular}{ |p{4cm}|p{5.5cm}|p{6cm}|p{3.5cm}|  }
\hline
\multicolumn{4}{|c|}{Аналитиеская механика.} \\
\hline
Положение материальной точки:                                                &  % [1, 1]
$\vec{r} = x(t) \vec{i} + y(t) \vec{j} + z(t) \vec{k}$                       &  % [1, 2]
                                                                             &  % [1, 3]
                                                                             \\ % [1, 4]
\hline
Скорость материальной точки:                                                 &  % [2, 1]
$\vec{v} = \frac{d{\vec{r}}}{dt} =
\frac{dx}{dt} \vec{i} + \frac{dy}{dt} \vec{j} + \frac{dx}{dt} \vec{k}$       &  % [2, 2]
                                                                             &  % [2, 3]
                                                                             \\ % [2, 4]
\hline
Ускорение материальной точки:                                                &  % [3, 1]
$\vec{w} = \frac{d{\vec{v}}}{dt} =
\frac{d^2 x}{dt^2} \vec{i} +
\frac{d^2 y}{dt^2} \vec{j} + \frac{d^2 x}{dt^2} \vec{k}$                     &  % [2, 2]
                                                                             &  % [3, 3]
                                                                             \\ % [3, 4]
\hline
Вектор $\vec{\tau}$, определение:                                            &  % [4, 1]
$\begin{aligned}
\vec{v} = \frac{d}{dt} \vec{r[s(t)]} =                         
\frac{d\vec{r}}{ds} \frac{ds}{dt} = \vec{\tau} \frac{ds}{dt},   \\
\vec{\tau} = \frac{d\vec{r}}{ds}
\end{aligned}$                                                               &  % [4, 2]
                                                                             &  % [4, 3]
                                                                             \\ % [4, 4]
\hline
Ускорение через $\vec{\tau}$:                                                &  % [5, 1]
$\vec{w} = \frac{d\vec{v}}{dt} =
\frac{d}{dt} (v \vec{\tau}) =
\frac{dv}{dt} \vec{\tau} + v \frac{d\vec{\tau}}{ds} \frac{ds}{dt} =
\frac{dv}{dt} \vec{\tau} + v^2 \frac{d \vec{\tau}}{ds}$                      &  % [5, 2]
                                                                             &  % [5, 3]
                                                                             \\ % [5, 4]
\hline
Вектор кривизны, и его связь с $\vec{n}$:                                    &  % [6, 1]
$\frac{d\vec{\tau}}{ds} = \frac{1}{\rho} \vec{n}$                            &  % [6, 2]
                                                                             &  % [6, 3]
                                                                             \\ % [6, 4]
\hline
Разложение $\vec{w}$ по $\vec{\tau}$ и $\vec{n}$:                            &  % [7, 1]
$\vec{w} = \frac{dv}{dt} \vec{\tau} + \frac{v^2}{\rho} \vec{n}$              &  % [7, 2]
                                                                             &  % [7, 3]
                                                                             \\ % [7, 4]
\hline
Вектор бинормали $\vec{b}$ :                                                 &  % [8, 1]
$\vec{b} = \vec{\tau} \times \vec{n}$                                        &  % [8, 2]
                                                                             &  % [8, 3]
                                                                             \\ % [8, 4]
\hline
Касательные к координатныйм линиям ($\vec{r} = \vec{r} (q_1, q_2, q_3)$):    &  % [9, 1]
$\begin{aligned}
\frac{\partial \vec{r}}{\partial q_1} =
\frac{\partial x}{\partial q_1} \vec{i} + \frac{\partial y}{\partial q_1} \vec{j} +
\frac{\partial z}{\partial q_1} \vec{k} = H_1 \vec{e_1}, \\
\frac{\partial \vec{r}}{\partial q_2} =
\frac{\partial x}{\partial q_2} \vec{i} + \frac{\partial y}{\partial q_2} \vec{j} +
\frac{\partial z}{\partial q_2} \vec{k} = H_2 \vec{e_2}, \\
\frac{\partial \vec{r}}{\partial q_3} =
\frac{\partial x}{\partial q_3} \vec{i} + \frac{\partial y}{\partial q_3} \vec{j} +
\frac{\partial z}{\partial q_3} \vec{k} = H_3 \vec{e_3} 
\end{aligned}$                                                               &  % [9, 2]
                                                                             &  % [9, 3]
                                                                             \\ % [9, 4]
\hline
Коэффициенты Ляме:                                                           &  % [10, 1]
$H_k = \sqrt{(\frac{\partial x}{\partial q_k})^2 +
 (\frac{\partial y}{\partial q_k})^2 + (\frac{\partial z}{\partial q_k})^2}$ &  % [10, 2]
                                                                             &  % [10, 3]
                                                                             \\ % [10, 4]
\hline
Ортогональные криволинейные координаты:                                      &  % [11, 1]
$(\vec{e_1} \cdot \vec{e_2}) = (\vec{e_2} \cdot \vec{e_3}) =
 (\vec{e_3} \cdot \vec{e_1}) = 0 $                                           &  % [11, 2]
                                                                             &  % [11, 3]
                                                                             \\ % [11, 4]
\hline
Эквивалентные условия ортогональности криволинейных координат:               &  % [12, 1]
$\frac{\partial x}{\partial q_l} \frac{\partial x}{\partial q_m} =            
 \frac{\partial y}{\partial q_l} \frac{\partial y}{\partial q_m} =
 \frac{\partial z}{\partial q_l} \frac{\partial z}{\partial q_m} = 0$
для $l \neq m$                                                               &  % [12, 2]
                                                                             &  % [12, 3]
                                                                             \\ % [12, 4]
\hline
Дифференциал дуги произвольной кривой (метрика пространства):                &  % [13, 1]
$\begin{aligned}
ds^2 = dx^2 + dy^2 + dz^2 =                       \\
\sum_{i = 1}^3 \sum_{j = 1}^3 g_{ij} dq_i dq_j
\end{aligned}$                                                               &  % [13, 2]
                                                                             &  % [13, 3]
                                                                             \\ % [13, 4]
\hline
Метрика пространства (случай ортогональных координат):                       &  % [14, 1]
$ds^2 = H_1^2 d q_1^2 + H_2^2 d q_2^2 + H_3^2 d q_3^2$                           &  % [14, 2]
                                                                             &  % [14, 3]
                                                                             \\ % [14, 4]
\hline
Скорость через криволинейные координаты:                                     &  % [15, 1]
$\vec{v} = \frac{d\vec{r}}{dt} =
 \frac{\partial \vec{r}}{\partial q_1} \dot{q_1} +
 \frac{\partial \vec{r}}{\partial q_2} \dot{q_2} +
 \frac{\partial \vec{r}}{\partial q_3} \dot{q_3} = 
H_1 \dot q_1 \vec{e_1} + H_2 \dot q_2 \vec{e_2} + H_2 \dot q_2 \vec{e_2}$    &  % [15, 2]
                                                                             &  % [15, 3]
                                                                             \\ % [15, 4]
\hline
                                                                             &  % [16, 1]
                                                                             &  % [16, 2]
                                                                             &  % [16, 3]
                                                                             \\ % [16, 4]
\hline
                                                                             &  % [17, 1]
                                                                             &  % [17, 2]
                                                                             &  % [17, 3]
                                                                             \\ % [17, 4]
\hline
                                                                             &  % [18, 1]
                                                                             &  % [18, 2]
                                                                             &  % [18, 3]
                                                                             \\ % [18, 4]
\hline
                                                                             &  % [19, 1]
                                                                             &  % [19, 2]
                                                                             &  % [19, 3]
                                                                             \\ % [19, 4]
\hline
                                                                             &  % [20, 1]
                                                                             &  % [20, 2]
                                                                             &  % [20, 3]
                                                                             \\ % [20, 4]
\hline
                                                                             &  % [21, 1]
                                                                             &  % [21, 2]
                                                                             &  % [21, 3]
                                                                             \\ % [21, 4]
\hline
                                                                             &  % [22, 1]
                                                                             &  % [22, 2]
                                                                             &  % [22, 3]
                                                                             \\ % [22, 4]
\hline
                                                                             &  % [23, 1]
                                                                             &  % [23, 2]
                                                                             &  % [23, 3]
                                                                             \\ % [23, 4]
\hline
                                                                             &  % [24, 1]
                                                                             &  % [24, 2]
                                                                             &  % [24, 3]
                                                                             \\ % [24, 4]
\hline
                                                                             &  % [25, 1]
                                                                             &  % [25, 2]
                                                                             &  % [25, 3]
                                                                             \\ % [25, 4]
\hline
\end{tabular}

\newpage

\end{document}